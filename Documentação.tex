\documentclass{article}
\title {1\b o Trabalho de Algoritmos Num\'ericos}
\author{Matheus H. Risso e Pedro P. Ladeira}
\date{03/05/2016}
\begin{document}
\pagenumbering{gobble}
\maketitle
\begin{enviroment}
\tiny
Trabalho da disciplina de Algoritmos Numéricos I, ministrada pelo Dr. Thomas W. Rauber
\end {enviroment}
\newpage
\pagenumbering{arabic}
\renewcommand{\contentsname}{Sum\'ario}
\tableofcontents
\newpage

\section{Introdu\c c\~ao}
\paragraph{}Equa\c c\~ao diferencial ordin\'aria (EDO) \'e uma equa\c c\~ao que apresenta derivadas ou diferenciais de uma fun\c c\~ao desconhecida que
possui apenas uma vari\'avel independente. 
\paragraph{}EDOs s\~ao frequentemente usadas na f\'isica para descrever fen\^omenos naturais, por\'em nem sempre uma EDO possui uma solu\c c\~ao anal\'itica. 
Para solucionar isso foram criados os m\'etodos num\'ericos que aproximam-se de uma solu\c c\~ao anal\'itica. 
\paragraph{}Iremos abordar neste trabalho fun\c c\~oes que possuem solu\c c\~oes anal\'iticas e aplicar os m\'etodos de aproxima\c c\~ao para solu\c c\~ao num\'erica
de Euler, Euler Melhorado, Euler Modificado, Runge-Kutta de terceira ordem e Bogacki-Shampine.
\newpage
\section{Objetivos}
\paragraph{}Este trabalho tem como objetivo entender e comparar diversos m\'etodos num\'ericos quanto a efici\^encia em se aproximar de solu\c c\~oes anal\'iticas e 
os aplicar num problema pr\'atico.
\paragraph{}Para isso utilizaremos a ferramenta computacional Octave GNU para criar fun\c c\~oes que descrevem os m\'etodos num\'ericos citados na especifica\c c\~ao.
Al\'em disso cuidaremos para que o c\'odigo seja compat\'ivel com Octave como com o MATLAB.

\newpage
\section{Metodologia}
	\subsection{Cria\c c\~ao dos m\'etodos}
\paragraph{}Utilizando o material sugerido na especifica\c c\~ao do trabalho entendemos cada m\'etodo e criamos fun\c c\~oes que descrevem cada um deles. Com isso 
comparamos com a solu\c c\~ao anal\'itica graficamente, definimos o erro e montamos uma tabela com os pontos.
		\subsubsection{Euler}
		\subsubsection{Euler Melhorado}
		\subsubsection{Euler Modificado}
		\subsubsection{Runge-Kutta de 3\b a Ordem}
		\subsubsection{Bogacki-Shampine}
		\subsubsection{Gr\'afico, C\'alculo do Erro e Tabela de Pontos}
	\subsection{Aplica\c c\~ao Pr\'atica}
\newpage
\section{Resultados e Avalia\c c\~oes}
\newpage
\section{Refer\^encias Bibliogr\'aficas}
http://www.somatematica.com.br/superior/equacoesdif/eq.php
https://pt.wikipedia.org/wiki/Lista_de_m%C3%A9todos_Runge-Kutta
https://pt.wikipedia.org/wiki/M%C3%A9todo_de_Runge-Kutta
\end{document}