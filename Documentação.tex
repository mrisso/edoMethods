\documentclass{article}
\usepackage{graphicx}
\begin{document}
\pagenumbering{gobble}
\begin{enviroment}
\end{enviroment}
\thispagestyle{empty}
\begin{center}
{\Large \bf Universidade Federal do Esp\'irito Santo\\[5pt]
Departamento de Inform\'atica
\end{center}
\vspace
*
{\stretch{1}}
\begin{center}
{\huge \bf 
1\b o Trabalho de Algoritmos Num\'ericos:\\
M\'etodos de aproxima\c c\~ao

\end{center}
\vspace
*
{\stretch{1}}
\begin{flushright}
Alunos: Matheus H. Risso e Pedro P. Ladeira\\
Professor: Prof. Dr. Thomas W. Rauber
\end{flushright}
\vspace
*
{\stretch{1}}
\begin{center}\begin{minipage}{12cm}
Trabalho da disciplina de Algoritmos Num\'ericos I, 
ministrada pelo Professor Dr. Thomas W. Rauber como forma de avalia\c c\~ao;
Universidade Federal do Esp\'irito Santo, 2016/1.
\end{minipage}\end{center}
\vspace
*
{\stretch{1}}
\centerline{\bf Vit\'oria-ES, 3 de Maio de 2016}
\vspace
*
{\stretch{1}}

\newpage
\pagenumbering{arabic}
\renewcommand{\contentsname}{Sum\'ario}
\tableofcontents

\newpage

\section{Introdu\c c\~ao}
\paragraph{}Equa\c c\~ao diferencial ordin\'aria (EDO) \'e uma equa\c c\~ao que apresenta derivadas ou diferenciais de uma fun\c c\~ao desconhecida que
possui apenas uma vari\'avel independente. 
\paragraph{}EDOs s\~ao frequentemente usadas na f\'isica para descrever fen\^omenos naturais, por\'em nem sempre uma EDO possui uma solu\c c\~ao anal\'itica. 
Para solucionar isso foram criados os m\'etodos num\'ericos que aproximam-se de uma solu\c c\~ao anal\'itica. 
\paragraph{}Iremos abordar neste trabalho fun\c c\~oes que possuem solu\c c\~oes anal\'iticas e aplicar os m\'etodos de aproxima\c c\~ao para solu\c c\~ao num\'erica
de Euler, Euler Melhorado, Euler Modificado, Runge-Kutta de terceira ordem (expl\'icito) e Dormand-Prince (embutido).
\newpage
\section{Objetivos}
\paragraph{}Este trabalho tem como objetivo entender e comparar diversos m\'etodos num\'ericos quanto a efici\^encia em se aproximar de solu\c c\~oes anal\'iticas e 
os aplicar num problema pr\'atico.
\paragraph{}Para isso utilizaremos a ferramenta computacional Octave GNU para criar fun\c c\~oes que descrevem os m\'etodos num\'ericos citados na especifica\c c\~ao.
Al\'em disso cuidaremos para que o c\'odigo seja compat\'ivel tanto com Octave como com o MATLAB.

\newpage
\section{Metodologia}
	\subsection{Cria\c c\~ao das fun\c c\~oes descritivas}
	\paragraph{}Utilizando o material sugerido na especifica\c c\~ao do trabalho entendemos cada m\'etodo e criamos fun\c c\~oes que descrevem cada um deles. Com isso 
	comparamos com a solu\c c\~ao anal\'itica graficamente, definimos o erro e montamos uma tabela com os pontos.
		\subsubsection{Euler}
		C\'odigo fonte que descreve o m\'etodo de Euler, dado a fun\c c\~ao(func), limite inferior de x(xi), 
		y inicial(yi), quantidade de pontos(p) e limite superior x(xf).
			\begin{figure}[h!]
			\includegraphics[width=\linewidth]{Euler.jpg}
			\caption{M\'etodo de Euler}
			\end{figure}
		\subsubsection{Euler Melhorado}
		C\'odigo fonte que descreve o m\'etodo de Euler Melhorado, dado a fun\c c\~ao(func), limite inferior de x(xi), 
		y inicial(yi), quantidade de pontos(p) e limite superior x(xf).
			\begin{figure}[h!]
			\includegraphics[width=\linewidth]{EulerMelhorado.jpg}
			\caption{M\'etodo de Euler Melhorado}
			\end{figure}
\newpage
		\subsubsection{Euler Modificado}
		C\'odigo fonte que descreve o m\'etodo de Euler Modificado, dado a fun\c c\~ao(func), limite inferior de x(xi), 
		y inicial(yi), quantidade de pontos(p) e limite superior x(xf).
			\begin{figure}[h!]
			\includegraphics[width=\linewidth]{EulerModificado.jpg}
			\caption{M\'etodo de Euler Modificado}
			\end{figure}
		\subsubsection{Runge-Kutta de 3\b a Ordem}
		C\'odigo fonte que descreve o m\'etodo de Runge-Kutta de 3\b a Ordem, dado a fun\c c\~ao(func), limite inferior de x(xi), 
		y inicial(yi), quantidade de pontos(p) e limite superior x(xf).
			\begin{figure}[h!]
			\includegraphics[width=\linewidth]{RungeKutta.jpg}
			\caption{M\'etodo de Euler Runge-Kutta de 3\b a Ordem}
			\end{figure}
\newpage
		\subsubsection{Dormand-Prince}
		C\'odigo fonte que descreve o m\'etodo de Dormand-Prince, dado a fun\c c\~ao(func), limite inferior de x(xi), 
		y inicial(yi), quantidade de pontos(p) e limite superior x(xf).
			\begin{figure}[h!]
			\includegraphics[width=\linewidth]{DormandPrince.jpg}
			\caption{M\'etodo de Dormand-Prince}
			\end{figure}
		\subsubsection{Gr\'afico Comparativo, C\'alculo de Erro e Tabela}
		Criamos uma fun\c c\~ao para plotar os m\'etodos em um gr\'afico afim de compara-los com a solu\c c\~ao anal\'itica. 
		Al\'em disso esta fun\c c\~ao mostra o erro graficamente como uma linha que liga os pontos de cada m\'etodo 
		que possuem a mesma ordenada e gera uma tabela dos pontos.
		\begin{itemize}
		\item C\'odigo-Fonte
		\item Gr\'afico
		\item Tabela
		\end{itemize}
	\subsection{Aplica\c c\~ao Pr\'atica (Problema do Reservat\'orio) }
\newpage
\section{Resultados e Avalia\c c\~oes}
\newpage
\section{Refer\^encias Bibliogr\'aficas}
\begin{itemize}
\item
http://www.somatematica.com.br/superior/equacoesdif/eq.php
\item
https://pt.wikipedia.org/wiki/Lista\_de\_m\%C3\%A9todos\_Runge-Kutta
\item
https://pt.wikipedia.org/wiki/M\%C3\%A9todo\_de\_Runge-Kutta
\item 
CAMPOS FILHO, Frederico Ferreira. Algoritmos Num\'ericos. 2 ed. Rio de Janeiro: LTC, 2012.
\end{itemize}
\newpage
\end{document}